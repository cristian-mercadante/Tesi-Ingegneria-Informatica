\chapter{Configurazione}
\label{Configurazione}
\thispagestyle{plain}

\section{Primo avvio}
Una volta eseguito Docker Compose, il sistema risulta avviato, funzionante e raggiungibile sulla porta 80, dall'host all'indirizzo \verb|localhost|, oppure da una macchina collegata alla rete locale tramite \verb|hostname|. Per quanto riguarda l'installazione fatta in ateneo, la macchine è raggiungibile all'indirizzo \verb|sharelatex.ing.unimore.it|, che verrà preso come esempio per i passi successivi.

\subsection{Registrazione dell'amministratore}
\begin{figure}[h]
    \begin{subfigure}{0.5\textwidth}
        \centering
        \includegraphics[height=6cm]{immagini/launchpad_1.png}
        \caption{Launchpad}
        \label{fig:sharelatex_launchpad_1}
    \end{subfigure}
    \begin{subfigure}{0.5\textwidth}
        \centering
        \includegraphics[height=6cm]{immagini/launchpad_2.png}
        \caption{Schermata di controllo}
        \label{fig:sharelatex_launchpad_2}
    \end{subfigure}
    \caption{Creazione utente amministratore}
    \label{fig:sharelatex_launchpad}
\end{figure}
\noindent Per registrare l'amministratore del sistema, visitare la pagina all'indirizzo\\\verb|sharelatex.ing.unimore.it/launchpad|. Una volta inseriti indirizzo e-mail e password dell'amministratore, verrà richiesto il primo login. Inserite le credenziali dell'amministratore, verrà mostrato il launchpad, il quale mostra lo stato dell'editor e delle websocket e dal quale è possibile inviare un'email di test a un indirizzo a scelta\footnote{Per usufruire di questa funzionalità è necessario configurare il servizio mail con \hyperref[SMTP]{SMTP}.}.

\subsection{Gestione utenti}
L'amministratore può registrare utenti normali all'applicazione aggiungendo la loro email nella pagina \enquote*{Manage Users}. La piattaforma genera un link per impostare la password del nuovo account.

Per eliminare un utente e tutti i progetti salvati sul suo account è invece necessario eseguire il seguente comando, indicando l'indirizzo mail dell'account da eliminare:
\begin{lstlisting}
sudo docker exec sharelatex /bin/bash -c "cd /var/www/sharelatex; grunt user:delete --email user@sharelatex.com"
\end{lstlisting}

\section{Interfaccia}
\verb|docker-compose.yml| permette anche di impostare variabili d'ambiente all'interno dei container. ShareLaTeX è altamente configurabile sotto questo aspetto. Si elencano le variabili d'ambiente utilizzate per l'installazione in ateneo:
\begin{itemize}
    \item \verb|SHARELATEX_APP_NAME|: nome dell'applicazione, che sarà mostrato nella barra di navigazione del browser.
    \item \verb|SHARELATEX_NAV_TITLE|: nome dell'applicazione, che sarà mostrato in alto a sinistra nell'homepage del profilo utente.
    \item \verb|SHARELATEX_HEADER_IMAGE|: URL di un'immagine, che sarà mostrata in alto a sinistra. Nel caso in cui \verb|SHARELATEX_NAV_TITLE| fosse impostato, allora\\\verb|SHARELATEX_HEADER_IMAGE| avrà la priorità.
    \item \verb|SHARELATEX_SITE_URL|: indirizzo della macchina host del servizio.
    \item \verb|SHARELATEX_ADMIN_EMAIL|: indirizzo e-mail dell'amministratore di sistema, che sarà mostrata in fase di registrazione utente.
    \item \verb|SHARELATEX_LEFT_FOOTER|: array JSON in cui si può aggiungere testo HTML che verrà mostrato in basso a sinistra nell'homepage.
    \item \verb|SHARELATEX_RIGHT_FOOTER|: array JSON in cui si può aggiungere testo HTML che verrà mostrato in basso a destra nell'homepage.
    \item \verb|SHARELATEX_SITE_LANGUAGE|: lingua dell'applicazione. Di default la lingua impostata è inglese, ma sono disponibili anche:
        \begin{multicols}{3}
            \begin{itemize}
                \item es = spagnolo
                \item pt = portoghese
                \item de = tedesco
                \item fr = francese
                \item cs = ceco
                \item nl = olandese
                \item sv = svedese
                \item it = italiano
                \item tr = turco
                \item cn = cinese
                \item no = norvegese
                \item da = danese
                \item ru = russo
                \item ko = coreano
                \item ja = giapponese
            \end{itemize}
        \end{multicols}
\end{itemize}

\section{SMTP}
\label{SMTP}
In fase di registrazione di un nuovo utente, come si è già detto, viene generato un link di iscrizione. Tale link deve essere in qualche modo consegnato al nuovo utente. Se il servizio email è configurato, allora questa fase sarà automatica. I parametri sono passati mediante variabili d'ambiente. Si raccomanda di impostare \verb|SHARELATEX_SITE_URL| per generare link di iscrizione funzionanti. Si elencano le variabili d'ambiente utilizzate per l'installazione in ateneo:
\begin{itemize}
    \item \verb|SHARELATEX_EMAIL_FROM_ADDRESS|: indirizzo del mittente del messaggio.
    \item \verb|SHARELATEX_EMAIL_SMTP_HOST|: host del servizio SMTP.
    \item \verb|SHARELATEX_EMAIL_SMTP_PORT|: porta SMTP da utilizzare.
    \item \verb|SHARELATEX_EMAIL_SMTP_SECURE|: valore booleano che, se vero, attiva SMTPS.
    \item \verb|SHARELATEX_EMAIL_SMTP_USER|: nome utente da autenticare.
    \item \verb|SHARELATEX_EMAIL_STMP_PASS|: password dell'utente da autenticare.
    \item \verb|SHARELATEX_EMAIL_STMP_TLS_REJECT_UNAUTH|: valore booleano che, se vero, rifiuta connessioni TLS (Transport Layer Security) non autorizzate.
    \item \verb|SHARELATEX_EMAIL_STMP_IGNORE_TLS|: valore booleano che, se vero, disattiva il supporto STARTTLS.
    \item \verb|SHARELATEX_CUSTOM_EMAIL_FOOTER|: testo HTML con cui si può personalizzare il footer delle email.
\end{itemize}