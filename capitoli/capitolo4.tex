\chapter{Installazione}
\label{Installazione}
\thispagestyle{empty}

\lstset{style=my-bash}

\section{Docker}

\subsection{Download}
Aggiornare l'indice di \verb|apt-get|.
\begin{lstlisting}
sudo apt-get update
\end{lstlisting}
Installare i pacchetti per permettere a \verb|apt-get| di utilizzare un repository su HTTPS.
\begin{lstlisting}
sudo apt-get install \ 
    apt-transport-https \
    ca-certificates \
    curl \
    software-properties-common
\end{lstlisting}
Aggiungere chiave GPG ufficiale di Docker.
\begin{lstlisting}
curl -fsSL https://download.docker.com/linux/ubuntu/gpg | \
    sudo apt-key add -
\end{lstlisting}
Verificare di possedere la chiave con il fingerprint\\
\verb|9DC8 5822 9FC7 DD38 854A E2D8 8D81 803C 0EBF CD88|\\
cercando gli ultimi 8 caratteri della fingerprint.
\begin{lstlisting}
sudo apt-key fingerprint 0EBFCD88
\end{lstlisting}
L'output dovrebbe essere:
\begin{lstlisting}
pub   4096R/0EBFCD88 2017-02-22
Key fingerprint = 9DC8 5822 9FC7 DD38 854A \
                    E2D8 8D81 803C 0EBF CD88
uid     Docker Release (CE deb) <docker@docker.com>
sub     4096R/F273FCD8 2017-02-22
\end{lstlisting}
Impostare il repository stabile.
\begin{lstlisting}
sudo add-apt-repository \
"deb [arch=amd64] https://download.docker.com/linux/ubuntu \
$(lsb_release -cs) \
stable"
\end{lstlisting}

\subsection{Installazione}
Installare l'ultima versione di Docker CE.
\begin{lstlisting}
sudo apt-get install -y docker-ce
\end{lstlisting}
Verificare che Docker CE sia installato correttamente avviando l'immagine \verb|hello-world|.
\begin{lstlisting}
sudo docker run hello-world
\end{lstlisting}


\section{Docker Compose}
\subsection{Download e installazione}
Download Docker Compose.
\begin{lstlisting}
sudo curl -L \
    https://github.com/docker/compose/releases/download/\
1.21.2/docker-compose-$(uname -s)-$(uname -m) \
    -o /usr/local/bin/docker-compose
\end{lstlisting}
Applicare i permessi di esecuzione al binario.
\begin{lstlisting}
sudo chmod +x /usr/local/bin/docker-compose
\end{lstlisting}
Testare l'installazione.
\begin{lstlisting}
docker-compose --version
\end{lstlisting}

\subsection{Configurazione}
Creare una directory dal nome sharelatex e inserire \verb|docker-compose.yml|. Oltre a creare il container di ShareLaTeX, il file prepara Redis e MongoDB. Inoltre fornisce una buona interfaccia per fornire impostazioni a ShareLaTeX.
\begin{lstlisting}
mkdir sharelatex
cd sharelatex
wget https://raw.githubusercontent.com/cristian-mercadante/\
sharelatex/master/docker-compose.yml
\end{lstlisting}
Infine avviare i container di MongoDB, Redis e ShareLaTeX usando le impostazioni in \verb|docker-compose.yml|.
\begin{lstlisting}
sudo docker-compose up -d
\end{lstlisting}

