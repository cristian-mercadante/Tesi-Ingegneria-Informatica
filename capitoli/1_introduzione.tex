\chapter{Introduzione}
\label{Introduzione}
\thispagestyle{empty}

Il seguente elaborato descrive le fasi di installazione e configurazione del software \emph{ShareLaTeX} su una macchina virtuale ospitata da un server all'interno del Dipartimento di Ingegneria Enzo Ferrari dell'Università di Modena e Reggio Emilia.

\section{Motivazioni}
L'editor di testo ShareLaTeX è assai utilizzato all'interno del gruppo di ricerca del dipartimento. Professori, dottori e dottorandi utilizzano questa piattaforma per scrivere, memorizzare e aggiornare le proprie pubblicazioni scientifiche ed ingegneristiche scritte in \LaTeX. Il gruppo di ricerca AImageLab necessitava di un'installazione locale di ShareLaTeX per far fronte a due necessità:
\begin{itemize}
    \item Evitare problemi di disconnessione che possano compromettere lo stato e la modifica dei files. Mediante connessione in rete locale al servizio vengono scongiurati problemi di questo tipo.
    \item Conservare progetti e documenti sul server dell'ateneo in modo sicuro.
\end{itemize}
Pertanto si è pensato di sfruttare le guide e i tools messi a disposizione dagli sviluppatori per installare un instanza locale di ShareLaTeX all'interno dell'ateneo. Si è poi deciso per ragioni di comodità di rendere accessibile il sito anche all'esterno dell'ateneo all'indirizzo \verb|sharelatex.ing.unimore.it|.

\section{Obiettivi}
L'obiettivo di questo progetto è installazione di ShareLaTeX e la sua configurazione per l'università. Inoltre si vuole che il sistema sia facilmente manutenibile in fase di aggiornamento dei suoi componenti.

\section{Strumenti}
Al fine dell'installazione, gli sviluppatori di ShareLaTeX hanno fornito un'immagine funzionante e aggiornata del sistema sulla piattaforma Docker. In seguito si parlerà delle sue funzionalità. È disponibile la guida ufficiale d'installazione\\(\verb|github.com/sharelatex/sharelatex/wiki|). Questa presenta sia una "Quick Start Guide" che paragrafi dettagliati sulle varie fasi d'installazione. Si è deciso di scrivere una guida analoga per l'installazione all'interno dell'ateneo in lingua italiana\\(\verb|github.com/aimagelab/sharelatex/wiki|).

\section{Contenuti}
\huge [FARE]
\normalsize

