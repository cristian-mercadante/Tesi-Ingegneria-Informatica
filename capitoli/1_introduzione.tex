\chapter{Introduzione}
\label{Introduzione}
\thispagestyle{empty}

Il seguente elaborato descrive le fasi di installazione e configurazione del software \emph{ShareLaTeX} su una macchina virtuale ospitata da un server all'interno del Dipartimento di Ingegneria Enzo Ferrari dell'Università di Modena e Reggio Emilia.

\section{Motivazioni}
L'applicazione ShareLaTeX è assai utilizzata all'interno dei gruppi di ricerca del dipartimento. Professori, ricercatori e dottorandi utilizzano quest'applicazione per scrivere, memorizzare e aggiornare le proprie pubblicazioni scientifiche scritte in \LaTeX. Il gruppo di ricerca AImageLab necessita di un'installazione di ShareLaTeX su un server locale, per far fronte a due esigenze:
\begin{itemize}
    \item Evitare problemi di disconnessione che possano compromettere lo stato e la modifica dei documenti. Mediante connessione in rete locale al servizio vengono scongiurati problemi di questo tipo.
    \item Conservare progetti e documenti sul server dell'ateneo in modo sicuro.
\end{itemize}
Pertanto si è pensato di sfruttare le guide e i tool messi a disposizione dagli sviluppatori dell'applicazione per installare un'istanza locale di ShareLaTeX all'interno dell'ateneo. Si è poi deciso, per permettere agli utenti di utilizzare il servizio anche al di fuori della rete universitaria, di rendere accessibile il sito anche all'esterno dell'ateneo all'indirizzo \url{sharelatex.ing.unimore.it}.

\section{Obiettivi}
L'obiettivo di questo progetto è l'installazione di ShareLaTeX e la sua configurazione per il Dipartimento di Ingegneria Enzo Ferrari. Inoltre si vuole che il sistema sia facilmente manutenibile in fase di aggiornamento dei suoi componenti.

\section{Strumenti}
Al fine dell'installazione, gli sviluppatori di ShareLaTeX hanno fornito un'immagine funzionante e aggiornata del sistema sulla piattaforma Docker. È disponibile la guida ufficiale d'installazione \cite{sharelatex_wiki}, che presenta sia una \enquote*{Quick Start Guide} \cite{sharelatex_qsg}, sia paragrafi dettagliati sulle varie fasi d'installazione.

\section{Contenuti}
Si presenta ora un riassunto di ciò che sarà affrontato nei capitoli successivi. I capitoli 2 e 3 affronteranno aspetti tecnici e teorici, mentre i capitoli 4, 5, 6 e 7 affronteranno aspetti pratici, legati all'installazione di ShareLaTeX.
\begin{itemize}
    \item Nel capitolo 2 si parlerà di ShareLaTeX, delle sue funzionalità e delle sue potenzialità.
    \item Nel capitolo 3 si parlerà della piattaforma Docker, dei suoi componenti, del suo funzionamento e delle sue potenzialità.
    \item Nel capitolo 4 si spiegherà come installare Docker, il tool Docker Compose e come installare pacchetti \TeX.
    \item Nel capitolo 5 si spiegherà come personalizzare e configurare ShareLaTeX.
    \item Nel capitolo 6 si spiegherà come eseguire backup, ripristino e aggiornamento del sistema.
    \item Nel capitolo 7 si concluderà la descrizione del progetto.
\end{itemize}