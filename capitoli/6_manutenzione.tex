\chapter{Manutenzione}
\label{Manutenzione}
\thispagestyle{empty}

In questa capitolo si spiega come provvedere a conservare lo stato dell'installazione locale di ShareLaTeX. È necessario pianificare le attività di ripristino dei dati del sistema in quanto l'avvio dei container con immagini aggiornate potrebbe impedire il corretto funzionamento del sistema o la perdita di dati e documenti finora creati. Si ricorda inoltre che un semplice riavvio dei container comporta il loro ripristino.

\section{Backup e ripristino}
Il \href{code:docker-compose.yml}{template} di \verb|docker-compose.yml| mostrato nel capitolo Installazione possiede già impostazioni riguardanti la persistenza dei dati dei tre container installati. Saranno aggiunte altre impostazioni per rendere più sicura la procedura di aggiornamento del sistema in futuro.

\subsection{MongoDB}
MongoDB possiede due tool chiamati \verb|mongodump| e \verb|mongorestore| che saranno utilizzati per esportare ed importare il dataset in un formato sicuro per il backup. Per ulteriori informazioni visitare la documentazione ufficiale di MongoDB all'indirizzo \url{https://docs.mongodb.com/manual/tutorial/backup-and-restore-tools/}. Innanzitutto eseguire \verb|mongodump|.
\begin{lstlisting}
sudo docker exec mongo mongodump
\end{lstlisting}
L'output sarà all'interno del container MongoDB in \verb|/dump|. Per rendere questi file persistenti occorre montare un volume esterno. Si noti che Il template di \verb|docker-compose.yml| crea di default un volume per la persistenza dei dati, che non è sufficiente in caso di upgrade o downgrade di MongoDB.
\begin{lstlisting}
volumes:
    - ~/mongo_data:/data/db
    - ~/mongodump_data:/dump
\end{lstlisting}
Riavviando il container l'output di \verb|mongodump| sarà accessibile nella home directory dell'host. Con \verb|mongorestore| sarà poi possibile ripristinare lo stato di MongoDB.
\begin{lstlisting}
sudo docker exec mongo mongorestore /dump
\end{lstlisting}

\subsection{Redis}
Redis contiene dati a breve termine, ma può essere comunque importante eseguire un backup dei dati. Il template di \verb|docker-compose.yml| crea di default un volume per la persistenza dei dati. Conterrà un unico file nominato \verb|dump.rdb|.
\begin{lstlisting}
volumes:
    - ~/redis_data:/data
\end{lstlisting}

\subsection{ShareLaTeX}
I dati relativi a ShareLaTeX sono salvati all'interno del container in \verb|/var/lib/sharelatex| e il template di \verb|docker-compose.yml| crea di default un volume per la persistenza dei dati.
\begin{lstlisting}
volumes:
    - ~/sharelatex_data:/var/lib/sharelatex
\end{lstlisting}

\section{Aggiornamento}
Una volta eseguiti i passi per gestire l'operazione di backup e ripristino è necessario definire gli step per eseguire l'upgrade (o downgrade) dei container in modo sicuro, senza perdita di dati.

\subsection{Lougout degli utenti}
È opportuno impedire agli utenti di utilizzare l'applicazione durante la fase di aggiornamento. Per far ciò l'amministratore deve fare login e accedere al pannello di gestione del sito presso \verb|sharelatex.ing.unimore.it/admin|.
\begin{figure}[h]
    \centering
    \includegraphics[width=\textwidth]{immagini/close_editor.PNG}
    \caption{Pannello di controllo dell'amministratore}
    \label{fig:close_editor}
\end{figure}
\begin{enumerate}
    \item Accedere alla scheda \enquote*{Close Editor}.
    \item Cliccare sul bottone \enquote*{Close Editor}: impedisce agli utenti di caricare l'editor di progetto.
    \item Cliccare sul bottone \enquote*{Disconnect all users}: chiunque sia connessio all'editor verrà rinviato ad una pagina di avviso manutenzione.
\end{enumerate}
L'unico modo per riavviare l'editor consiste nel riavvio dei container.

\subsection{Esecuzione di mongodump}
Per eseguire il backup dei dati di MongoDB bisogna eseguire \verb|mongodump|. Non sarà necessario conservare la directory \verb|mongo_data| in quanto \verb|mongorestore| sarà in grado di ripristinarla a dovere. L'output di \verb|mongodump| avverrà in \verb|/dump| all'interno del container. Avendo montato un volume in \verb|~/mongodump_data|, tale percorso sarà accessibile dall'esterno del container e persistente.
\begin{lstlisting}
sudo docker exec mongo mongodump
\end{lstlisting}

\subsection{Sospensione e rimozione dei container}
Si può scegliere se agire su tutti e tre i container o solo sul container da aggiornare.
\begin{lstlisting}
sudo docker stop sharelatex mongo redis
sudo docker rm sharelatex mongo redis
\end{lstlisting}

\subsection{Rimozione dei dati precedenti}
Eliminare la directory sull'host \verb|mongo_data|. È necessario in quanto i container possono rifiutare l'avvio con dati diversi: è il caso di MongoDB. Non bisogna eliminare invece la directory sull'host \verb|mongodump_data| perché frutto del montaggio di un volume esterno necessario per eseguire \verb|mongorestore|.
\begin{lstlisting}
sudo rm -r ~/mongo_data
\end{lstlisting}
Se al successivo riavvio del sistema il container Redis dovesse rifiutare il riavvio, aggiungere agli step precedenti l'eliminazione della diretory sull'host \verb|redis_data|.
\begin{lstlisting}
sudo rm -r ~/redis_data
\end{lstlisting}

\subsection{Riavvio del sistema}
Il template di \verb|docker-compose.yml| imposta come immagine di default per ogni container quella con tag \verb|latest|. È possibile anche selezionare una specifica immagine da scaricare. L'elenco delle immagini disponibili è su Docker Hub. Ad esempio, se si vuole l'immagine \verb|v1.0.0| di ShareLaTeX è necessario appendere il tag selezionato nel campo \verb|image| mediante \verb|:|. Segue un esempio di quanto detto.
\begin{lstlisting}
services:
    sharelatex:
        restart: always
        image: sharelatex/sharelatex:v1.0.0
        container_name: sharelatex
\end{lstlisting}
Quindi riavviare i container.
\begin{lstlisting}
sudo docker-compose up -d
\end{lstlisting}

\subsection{Ripristino con mongorestore}
Ci si presenterà una nuova installazione dell'applicazione. Per ripristinare i dati precedenti occorre eseguire \verb|mongorestore|.
\begin{lstlisting}
sudo docker exec mongo mongorestore /dump
\end{lstlisting}

\section{Automatizzazione degli aggiornamenti: update.sh}
Per automatizzare il processo di aggiornamento dei container è stato prodotto lo script \verb|update.sh|. Questo script esegue il procedimento precedentemente descritto passo per passo, mostrando all'utente lo stato di ogni fase. Per eseguire lo script è necessario scaricarlo e ottenere i permessi di esecuzione.
\begin{lstlisting}
wget https://raw.githubusercontent.com/aimagelab/sharelatex/master/update.sh
chmod +x update.sh
\end{lstlisting}
Lo script accetta un parametro che identifica una directory. Tale directory deve contenere un file \verb|docker-compose.yml| da utilizzare per avviare e configurare i container.
\begin{lstlisting}
# Utilizzo
./update.sh <dir>
\end{lstlisting}
Lo script controlla innanzitutto il numero dei parametri: se non sufficiente, allora lo notifica all'utente. Esegue quindi un controllo sul parametro, per verificare che identifichi una directory traversabile. Segue eseguendo \verb|mongodump|, arrestando i container, eliminando i dati superflui, riavviando i container, per poi ripristinare i dati con \verb|mongorestore|. Mostra infine lo stato attuale dei container attivi e chiede all'utente se proseguire con l'installazione completa (o aggiornamento) di TeX Live.

Si raccomanda di utilizzare questo script anche in caso di semplice riavvio dei container: dato che è stato impostato in \verb|docker-compose.yml| che l'immagine da utilizzare per la creazione dei container è quella con tag \enquote*{latest}, l'eventuale release di un'immagine aggiornata comporterebbe il suo download la sostituzione con l'immagine precedente, il che costituisce un vero e proprio aggiornamento.

Segue lo script \verb|update.sh|
\lstinputlisting[caption={Script update.sh}, captionpos=b, style=my-style]{script/update-ridotto.sh}



