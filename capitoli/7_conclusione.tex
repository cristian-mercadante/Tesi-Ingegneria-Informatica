\chapter{Conclusione}
\label{Conclusione}
\thispagestyle{plain}

Giunti a questo punto, il servizio è attivo e funzionante. Tutte le fasi precedentemente descritte sono state testate prima dell'installazione finale. Segue la precisazione di alcuni aspetti tralasciati nei precedenti capitoli.

\section{Funzionalità aggiuntive}
L'installazione effettuata rispecchia le necessità espresse nell'\hyperref[Introduzione]{capitolo introduttivo}. Alcune funzionalità aggiuntive sono rese disponibili dagli sviluppatori, ma non è stata ritenuta necessaria la loro installazione nel sistema in questione. Sono le seguenti.
\begin{itemize}
    \item Email tramite Amazon Web Services.
    \item Bilanciamento di carico con HAProxy.
    \item Restrizioni sulle password utente.
\end{itemize}
Queste ed altre informazioni sono disponibili sulla guida ufficiale \cite{sharelatex_wiki}.

\section{Requisiti hardware}
ShareLaTeX raccomanda un core CPU e 1 GB di RAM ogni 5 utenti connessi. Inoltre ShareLaTeX è un programma con un solo thread. L'operazione più onerosa è la compilazione di un documento che è quindi legata alla performance del singolo core, perciò maggiore sarà la velocità del singolo core, minore sarà il tempo richiesto alla compilazione.

\section{Capacità dei progetti in ShareLaTeX}
ShareLaTeX è in grado di gestire per ogni progetto fino a 5 MB di dati modificabili (senza includere immagini e altri file non modificabili). Ogni progetto può avere al massimo 2000 entità, ovvero cartelle, file, immagini, ecc. La dimensione massima per un file non modificabile (immagini, ecc.) è di 50 MB.