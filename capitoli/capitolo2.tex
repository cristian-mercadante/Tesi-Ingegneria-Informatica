\chapter{ShareLaTeX}
\label{ShareLaTeX}
\thispagestyle{empty}

ShareLaTeX è uno strumento assai diffuso in tutto il mondo all'interno della comunità scientifica e accademica. È un editor di documenti in linguaggio \LaTeX ~che presenta numerose funzionalità che lo caratterizzano rispetto ai normali editor.

\section{Funzionalità}
ShareLaTeX offre un'interfaccia utente semplice e immediata. Ogni utente iscritto possiede la sua homepage con i progetti creati, dalla quale può decidere se aprire un progetto esistente o crearne uno nuovo.

L'interfaccia dell'editor è molto semplice e divisa in tre parti:
\begin{itemize}
    \item Una colonna di navigazione fra i file del progetto, dalla quale si possono non solo creare nuovi file e cartelle, ma anche importarli da locale o dal web.
    \item La finestra di editor vera e propria, dove l'utente scrive testo e comandi \LaTeX, con evidenziazione della sintassi e consigli sul completamento dei comandi.
    \item La finestra di anteprima del documento, che presenta l'esito della compilazione.
\end{itemize}
ShareLaTeX è un editor collaborativo. Nel momento in cui il creatore di un documento non è una singola persona, ma un gruppo, risulta difficile sincronizzare i file e lo stato dei file con un normale editor di testo. ShareLaTeX consente la condivisione dei documenti con più persone, alle quali possono essere concessi sia privilegi di lettura che di scrittura. I file rimarranno consistenti e le modifiche apportate saranno visibili da parte di tutti gli utenti connessi in tempo reale.

I progetti conservano lo storico delle modifiche. È quindi possibile ripristinare ciascun file alla versione precedente, oppure ripristinare un file eliminato. Inoltre sarà possibile tenere traccia delle modifiche apportate al progetto e del relativo autore.

I progetti salvati su ShareLaTeX sono infine salvati sul server e accessibili da ovunque.

\section{Sviluppo e distribuzione}
ShareLaTeX è stato fondato da Henry Oswald e James Allen ed è guidato da un team di 8 persone. Possiede una licenza AGPL-3.0 e il codice sorgente e liberamente accessibile su GitHub (github.com/sharelatex). Nel corso degli anni la community ha contribuito alla crescita e al miglioramento della piattaforma.

Gli sviluppatori hanno fornito una guida per l'installazione e configurazione dei ShareLaTeX. Tale guida è stata usata come riferimento principale per lo svolgimento di questo progetto. Gli sviluppatori hanno inoltre fornito un immagine aggiornata e funzionante del sistema per la piattaforma Docker.

ShareLaTeX propone diversi piani di abbonamento all'utente, flessibili su base mensile e annuale, con prezzo dimezzato. I vari abbonamenti differiscono per il numero di collaboratori per progetto e per la presenza delle funzionalità aggiuntive di sincronizzazione con Dropbox e GitHub.

Il 20 luglio 2017 ShareLaTeX annunciò la fusione con Overleaf, un altro software per l'editing collaborativo di progetti LaTeX. Dall'unione nasce Overleaf v2. Dal 4 settembre 2018 ShareLaTeX non è più disponibile e tutti gli account creati sono stati esportati sulla nuova piattaforma.
Il team ha comunque dichiarato che ShareLaTeX continuerà a crescere e ad essere open source.